\documentclass{report}
\usepackage[utf8]{inputenc}

\title{Spiking Neural Networks, classifying Dynamic Vision Sensor Data }
\author{Georgios Alexakis,Dimitrios Korakobounis}
\date{2021}
\usepackage{cite}
\begin{document}



\maketitle

\tableofcontents{}
\chapter{Introduction}
     
\section{Thesis description}
Machine learning, a subset of AI, has grown in popularity in recent years, owing largely to advancements in GPU hardware and the massive amounts of data produced by the digital age. The concepts and algorithms used in the field today have been floating around for decades, but we have not been able to use them to their full extent until now. The majority of machine learning algorithms employ a simple artificial neural network structure consisting of multiple layers of interconnected neurons. In deep learning, a term used when the number of layers is large, each level learns to transform its input data into a slightly more abstract and composite representation. In image processing, for example, lower layers may identify edges, while higher layers may identify structures relevant to humans, such as numbers, letters, or faces.

However, while deep learning networks have advanced to the point that they outperform human performance in multiple tasks, the efficiency of these networks is orders of magnitude lower when compared to the human brain. Therefore it stands to reason to keep exploring the structure and inner workings of the human brain in order to increase the performance of machine learning algorithms and the hardware we use to implement them. 

This is where spiking neural networks come into play, neural networks that are far more inspired by information processing in biology than their predecessors(ANNs). The brain encodes information in sparse and asynchronous signals that are inherently processed in parallel. Deep learning neural networks process input layer by layer, and errors must be propagated backward in a non biologically plausible way. Processing information layer by layer indicates that information is not processed asynchronously. This limitation is imposed by the underlying hardware, synchronous circuits. A synchronous circuit is a digital circuit in digital electronics in which changes in the state of memory elements are synchronized by a clock signal. Learning methods in Spiking Neural Networks and a new type of computer hardware, neuromorphics, make an effort to utilize asynchronous processing.

For the reasons mentioned above, as well as the method by which we collect video data, using machine learning for video processing is one of the most computationally expensive tasks. Since standard cameras capture videos in image frames, the neural network must process all pixels each time a new frame is introduced.It appears to be much more efficient to be able to process the changing pixels asynchronously. This is why event cameras (Dynamic Vision Sensors) were developed. Event cameras are bio-inspired sensors that operate in a somewhat different way than conventional cameras. They calculate per-pixel brightness changes asynchronously rather than collecting images at a fixed time. As a result, a stream of events is produced that encodes the time, position, and sign of the brightness changes.

For the time being, these sensors are quite expensive, but thankfully, several DVS datasets have been recorded for researchers like us who want to test and develop spiking neural network machine learning algorithms in data obtained by these types of sensors.In addition, several python libraries have been introduced to handle spiking neural networks such as Bindsnet,NengoDL,pytorch-spiking,Norse and several others which are build on top of existing popular Machine Learning libraries, tensorflow and pytorch. We will be using Norse for our experiments.

\section{Spiking Neural Networks}
Testing citation   \cite{Esser2016}
Thesis description text
\section{Comparison with other Neural Networks}

Why Spiking Neural Networks? text

\section{Neuromorphic computing}

Neuromorphic computing text

\chapter{Brain Inspiration}
\section{Neurons}
\section{Dendrites and Synapses}
\section{Neural Representation of Information - Information Coding}
\section{Temporal Coding of Visual Space}
\section{Recurrence and Feedback in the Visual System}


\bibliographystyle{IEEEtran}
\bibliography{library}
\end{document}
